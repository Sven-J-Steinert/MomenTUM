\section{Individual Work - Julian Schmid}


%In this section the mass budget and the arrangement analysis script shall be covered in more detail together with the trade-off on the RCS configuration. Key aspects and simplifications of the scripts are pointed out so that the working principle is clear and can be set into perspective. The validation process of the mass budget script is analyzed to show, that it is a valid tool for the conceptional design process.


\subsection{Mass Budget Script and Validation}

The mass budget script is a program written in MATLAB which is mainly used for the preliminary sizing of the spacecraft. Its main functions are the following: 

\begin{enumerate}

   \item Estimation of the dry, wet and launch mass as well as the PWR subsystem needs and mass, STR subsystem mass and the tank masses and volumes

   \item Spacecraft mass behaviour over time
   
   \item Calculation of the engine characteristics together with the calculation of the mixed ISP between the main engine and the RCS thrusters and the engine thrust as a function of the inlet pressure
   
   \item Calculation of the main engine and RCS system propellant needs
   
   \item Calculation of the needed solar array area

\end{enumerate}

To make these calculations possible, the delta v's from the dV script as well as the already known subsystem masses and some key engine parameters have to be feed into the mass budget script. Two different approaches of calculating the spacecrafts wet mass are used and converged between each other to a residual of $10^{-3}$, to calculate the spacecrafts mass. This is needed, because the spacecrafts STR subsystem is directly dependent on the wet mass of the spacecraft itself. In this process, the ESA margin philosophy \cite{ESA.2012} is respected. Different loop-starters are used to calculate a spacecraft wet mass in the first iteration by adding up all subsystem masses. After that, the resulting dry mass out of this calculation is fed forward to a step-wise calculation which utilizes the Tsiolkovsky rocket equation for each maneuver \cite{Walter.2018} to calculate a second wet mass. This wet mass is then compared to the wet mass in the additive approach before. If these values are too far apart, the values of the calculation that uses the rocket equation, namely the propellant mass, are fed back into the additive approach until the convergence is reached. In a later stage, the propellant mass can directly be inserted from GMAT. 
For the estimation of the propulsion system, namely the tank masses, a database of components is generated, which can be found in the Appendix \ref{Appendix}. This can then be compared to the mass, the tank would have if only Barlow's formula would be used \cite{Manfletti.2022}. In the script, Barlow's formula is used with the combined average over all deviating tank masses from the formula as a margin. This leaves the propulsion system with the engine and tank masses respectively. By then calculating the relation, the tanks and the engine account for the total propulsion system in an already proposed missions, a margin can be determined. This margin also includes the piping and other not considered propulsion elements. The Ice Giants mission form ESA \cite{ESA.2019} for a bipropellant or the Comet Interceptor mission from ESA \cite{CDF.2019} for an electric system can be used as a reference. To calculate the engine performance, a linear interpolation is made between the minimum and maximum point of engine thrust at the dedicated minimum and maximum inlet pressure of the main-engine (ME) and the RCS thruster \cite{Ariane.2023}. This is shown in Equation \ref{eq:S400} for the S400-15 and Equation \ref{eq:S10} for the S10-26 respectively in $[\frac {N}{bar}]$.

\begin{equation}
F_{S400-15} = \frac {450-340}{18.5-12.5} \cdot (p_{tank}-p_{loss}) + 110.833
\label{eq:S400}
\end{equation}

\begin{equation}
F_{S10-26} =  \frac {12.5-6.0}{23.0-10.0} \cdot (p_{tank}-p_{loss}) + 1
\label{eq:S10}
\end{equation}

In terms of the main-engine efficiency, a mixed ISP is used between the main-engine and the RCS thrusters for the Tsiolkovsky rocket equation because two RCS thrusters have to compensate the shown engine misalignment and offset in Table \ref{tab:eng-mis}. In this case, the maximum thrust of all engines is considered. Equation \ref{eq:ISP} shows this process. 

\begin{equation}
ISP_{ME} = \frac {ISP_{ME} \cdot F_{ME} + 2 \cdot ISP_{RCS} \cdot F_{RCS}} {F_{ME} + 2 \cdot F_{RCS}}
\label{eq:ISP}
\end{equation}

In later mission planning phases, where multiple iterations between the mass budget and pressure drop script together with GMAT and the dV script occurred and a final flow-chart exists, the assumptions can be replaced by the chosen parts in the parts-list. For the PWR subsystem, the dedicated power requirements of each subsystem are inserted. By applying calculation logic, shown in the Section \ref{sec:solar}, the mass of this subsystem can be calculated. By following these formulas, the solar array area is computed. 
To see if the assumptions in this mass budget script are valid, the Ice Giants mission from ESA is used as a \href{https://github.com/Sven-J-Steinert/MomenTUM/blob/main/MATLAB/mass_design_icygigants.m}{\colorbox{codegray}{validation reference}} for this tool. In Table \ref{tab:valid} it can be seen, that the discrepancy in the wet mass is low but not below 1 \%. By tracing this error back it can be seen, that the biggest error lies in the propellant mass and not in the dry mass or the estimation of the propulsion system mass in general. One scenario where the divergence in the propellant mass can result from is, that the mass budget script does not accounting for the AOGNC maneuvers in the same way as the CDF report is. Since the mass budget script estimates a slightly higher wet mass, this error should be seen as an additional margin. For the estimation of the propulsion system mass as well as the dry mass, which is needed for the feedback into GMAT, this estimation is sufficient. Therefore, the mass budget script shall be used for the conceptual design process of this spacecraft.

\begin{table}[H]
\caption{Mass Budget Script Validation with ESA's Icy Giants CDF Report \cite{ESA.2019}}
\label{tab:valid}
\resizebox{\linewidth}{!}{%
\begin{tabular}{|lcc|}
\hline
\rowcolor[HTML]{8DB4E1} 
\multicolumn{3}{|c|}{\cellcolor[HTML]{8DB4E1}\textbf{Mass Budget Script Validation}}                                       \\ \hline
\rowcolor[HTML]{C5D9F0} 
\multicolumn{1}{|l|}{\cellcolor[HTML]{C5D9F0}}         & \multicolumn{1}{c|}{\cellcolor[HTML]{C5D9F0}CDF Report} & Script  \\ \hline
\multicolumn{1}{|l|}{Wet Mass {[}kg{]}}                & \multicolumn{1}{c|}{4398.33}                            & 4540.60 \\ \hline
\multicolumn{1}{|l|}{Wet Mass Error {[}\%{]}}          & \multicolumn{2}{c|}{3.13}                                         \\ \hline
\multicolumn{1}{|l|}{Propellant Mass Error {[}\%{]}}   & \multicolumn{2}{c|}{5.79}                                         \\ \hline
\multicolumn{1}{|l|}{Propulsion System Error {[}\%{]}} & \multicolumn{2}{c|}{0.74}                                         \\ \hline
\multicolumn{1}{|l|}{Dry Mass Error {[}\%{]}}          & \multicolumn{2}{c|}{0.21}                                         \\ \hline
\end{tabular}}
\end{table}



\subsection{RCS Configuration Analysis}

To decide, which RCS configuration shall be used for the mission, the following comparison is crated based on the work in the following paper \cite{Pasand.2017}. In this paper, 14 different configurations are created and analyzed in different categories. Because a configuration with less than 8 thrusters ensures no redundancy, only the configurations 6 to 14 are shown in Table \ref{tab:RCS-config}. By comparing the mean tracking error (MTE) in three axis as well as the level of redundancy (LR) it can be seen, that in configurations with 8 thrusters, only one thruster can fail to still ensure a three axis stabilized spacecraft. In systems with 16 thrusters, even 3 thrusters can fail. By calculating a ratio between the LR to the amount of thrusters it can be seen, that a systems with 8 provides a ratio of 0.125, a system with 12 roughly 0.166 and a system with 14 a ratio of 0.1875. This shows, that a system with 16 is only slightly more redundant as a system with 12 compared to its amount of thrusters, while the static fuel consumption and the average pulses per thruster, for the reference case in the paper, are in the same range. Therefore, it is argued, that a system with 16 thrusters is not as beneficial as a system with 12 if the 16 thruster system has a similar MTE in three axis, as well as in total a higher fuel consumption and more mass. The piping is also more complex, leading to a higher system mass overall. By then comparing the RCS configurations with 12 thrusters individually it can be seen, that the best MTE and the least amount of pulses can be achieved with configuration 10, while having a similar fuel consumption as concepts 11 and 12. Consequently, configuration 10 is chosen for the spacecraft. By investigating the trade-off between a cold-gas nitrogen (N2) and a bipropellant RCS system with MMH and MON-3, as demonstrated in Table \ref{tab:RCS-tradeoff}, it can be shown, that a system that uses bipropellant as its propellant is lighter in the chosen spacecraft configuration. The reference case for this trade-off is a mission, that only uses a bipropellant engine. This is not only due to the low ISP of the cold gas system but also since the main engine already uses bipropellant, a common feeding system can be utilized and therefore mass saved.

\begin{table}[H]
\caption{RCS configuration analysis \cite{Pasand.2017}}
\label{tab:RCS-config}
\resizebox{\linewidth}{!}{%
\begin{tabular}{|ccccc|}
\hline
\multicolumn{5}{|c|}{\cellcolor[HTML]{8DB4E1}\textbf{RCS configuration trade-off}}                                                                                                                                                                                                                                                                                                                                                                                                                                                                                                                                                               \\ \hline
\multicolumn{1}{|c|}{\cellcolor[HTML]{C5D9F0}}                                                                                          & \multicolumn{1}{c|}{\cellcolor[HTML]{C5D9F0}}                                                                                              & \multicolumn{1}{c|}{\cellcolor[HTML]{C5D9F0}}                                                                                                      & \multicolumn{1}{c|}{\cellcolor[HTML]{C5D9F0}}                                                                                                 & \cellcolor[HTML]{C5D9F0}                             \\
\multicolumn{1}{|c|}{\multirow{-2}{*}{\cellcolor[HTML]{C5D9F0}\begin{tabular}[c]{@{}c@{}}Config \\ (number of thrusters)\end{tabular}}} & \multicolumn{1}{c|}{\multirow{-2}{*}{\cellcolor[HTML]{C5D9F0}\begin{tabular}[c]{@{}c@{}}Static Fuel \\ Consumption {[}kg{]}\end{tabular}}} & \multicolumn{1}{c|}{\multirow{-2}{*}{\cellcolor[HTML]{C5D9F0}\begin{tabular}[c]{@{}c@{}}Mean Tracking \\ Error Three Axis {[}deg{]}\end{tabular}}} & \multicolumn{1}{c|}{\multirow{-2}{*}{\cellcolor[HTML]{C5D9F0}\begin{tabular}[c]{@{}c@{}}Average Pulses \\ per Thruster {[}1{]}\end{tabular}}} & \multirow{-2}{*}{\cellcolor[HTML]{C5D9F0}LR {[}1{]}} \\ \hline
\multicolumn{1}{|c|}{}                                                                                                                  & \multicolumn{1}{c|}{}                                                                                                                      & \multicolumn{1}{c|}{}                                                                                                                              & \multicolumn{1}{c|}{}                                                                                                                         &                                                      \\
\multicolumn{1}{|c|}{\multirow{-2}{*}{6 (8)}}                                                                                           & \multicolumn{1}{c|}{\multirow{-2}{*}{56.32}}                                                                                               & \multicolumn{1}{c|}{\multirow{-2}{*}{0.060}}                                                                                                       & \multicolumn{1}{c|}{\multirow{-2}{*}{289}}                                                                                                    & \multirow{-2}{*}{1}                                  \\ \hline
\multicolumn{1}{|c|}{}                                                                                                                  & \multicolumn{1}{c|}{}                                                                                                                      & \multicolumn{1}{c|}{}                                                                                                                              & \multicolumn{1}{c|}{}                                                                                                                         &                                                      \\
\multicolumn{1}{|c|}{\multirow{-2}{*}{7 (8)}}                                                                                           & \multicolumn{1}{c|}{\multirow{-2}{*}{50.54}}                                                                                               & \multicolumn{1}{c|}{\multirow{-2}{*}{0.057}}                                                                                                       & \multicolumn{1}{c|}{\multirow{-2}{*}{178}}                                                                                                    & \multirow{-2}{*}{1}                                  \\ \hline
\multicolumn{1}{|c|}{}                                                                                                                  & \multicolumn{1}{c|}{}                                                                                                                      & \multicolumn{1}{c|}{}                                                                                                                              & \multicolumn{1}{c|}{}                                                                                                                         &                                                      \\
\multicolumn{1}{|c|}{\multirow{-2}{*}{8 (8)}}                                                                                           & \multicolumn{1}{c|}{\multirow{-2}{*}{52.11}}                                                                                               & \multicolumn{1}{c|}{\multirow{-2}{*}{0.057}}                                                                                                       & \multicolumn{1}{c|}{\multirow{-2}{*}{199}}                                                                                                    & \multirow{-2}{*}{1}                                  \\ \hline
\multicolumn{1}{|c|}{}                                                                                                                  & \multicolumn{1}{c|}{}                                                                                                                      & \multicolumn{1}{c|}{}                                                                                                                              & \multicolumn{1}{c|}{}                                                                                                                         &                                                      \\
\multicolumn{1}{|c|}{\multirow{-2}{*}{9 (8)}}                                                                                           & \multicolumn{1}{c|}{\multirow{-2}{*}{79.40}}                                                                                               & \multicolumn{1}{c|}{\multirow{-2}{*}{0.220}}                                                                                                       & \multicolumn{1}{c|}{\multirow{-2}{*}{309}}                                                                                                    & \multirow{-2}{*}{1}                                  \\ \hline
\multicolumn{1}{|c|}{\cellcolor[HTML]{C5D9F0}}                                                                                          & \multicolumn{1}{c|}{\cellcolor[HTML]{C5D9F0}}                                                                                              & \multicolumn{1}{c|}{\cellcolor[HTML]{C5D9F0}}                                                                                                      & \multicolumn{1}{c|}{\cellcolor[HTML]{C5D9F0}}                                                                                                 & \cellcolor[HTML]{C5D9F0}                             \\
\multicolumn{1}{|c|}{\multirow{-2}{*}{\cellcolor[HTML]{C5D9F0}10 (12)}}                                                                 & \multicolumn{1}{c|}{\multirow{-2}{*}{\cellcolor[HTML]{C5D9F0}49.46}}                                                                       & \multicolumn{1}{c|}{\multirow{-2}{*}{\cellcolor[HTML]{C5D9F0}0.051}}                                                                               & \multicolumn{1}{c|}{\multirow{-2}{*}{\cellcolor[HTML]{C5D9F0}158}}                                                                            & \multirow{-2}{*}{\cellcolor[HTML]{C5D9F0}2}          \\ \hline
\multicolumn{1}{|c|}{}                                                                                                                  & \multicolumn{1}{c|}{}                                                                                                                      & \multicolumn{1}{c|}{}                                                                                                                              & \multicolumn{1}{c|}{}                                                                                                                         &                                                      \\
\multicolumn{1}{|c|}{\multirow{-2}{*}{11 (12)}}                                                                                         & \multicolumn{1}{c|}{\multirow{-2}{*}{50.81}}                                                                                               & \multicolumn{1}{c|}{\multirow{-2}{*}{0.077}}                                                                                                       & \multicolumn{1}{c|}{\multirow{-2}{*}{221}}                                                                                                    & \multirow{-2}{*}{2}                                  \\ \hline
\multicolumn{1}{|c|}{}                                                                                                                  & \multicolumn{1}{c|}{}                                                                                                                      & \multicolumn{1}{c|}{}                                                                                                                              & \multicolumn{1}{c|}{}                                                                                                                         &                                                      \\
\multicolumn{1}{|c|}{\multirow{-2}{*}{12 (12)}}                                                                                         & \multicolumn{1}{c|}{\multirow{-2}{*}{47.05}}                                                                                               & \multicolumn{1}{c|}{\multirow{-2}{*}{0.076}}                                                                                                       & \multicolumn{1}{c|}{\multirow{-2}{*}{224}}                                                                                                    & \multirow{-2}{*}{2}                                  \\ \hline
\multicolumn{1}{|c|}{}                                                                                                                  & \multicolumn{1}{c|}{}                                                                                                                      & \multicolumn{1}{c|}{}                                                                                                                              & \multicolumn{1}{c|}{}                                                                                                                         &                                                      \\
\multicolumn{1}{|c|}{\multirow{-2}{*}{13 (16)}}                                                                                         & \multicolumn{1}{c|}{\multirow{-2}{*}{46.49}}                                                                                               & \multicolumn{1}{c|}{\multirow{-2}{*}{0.047}}                                                                                                       & \multicolumn{1}{c|}{\multirow{-2}{*}{174}}                                                                                                    & \multirow{-2}{*}{3}                                  \\ \hline
\multicolumn{1}{|c|}{}                                                                                                                  & \multicolumn{1}{c|}{}                                                                                                                      & \multicolumn{1}{c|}{}                                                                                                                              & \multicolumn{1}{c|}{}                                                                                                                         &                                                      \\
\multicolumn{1}{|c|}{\multirow{-2}{*}{14 (16)}}                                                                                         & \multicolumn{1}{c|}{\multirow{-2}{*}{47.05}}                                                                                               & \multicolumn{1}{c|}{\multirow{-2}{*}{0.046}}                                                                                                       & \multicolumn{1}{c|}{\multirow{-2}{*}{188}}                                                                                                    & \multirow{-2}{*}{3}                                  \\ \hline
\end{tabular}}
\end{table}

\begin{table}[H]
\caption{RCS Propellant Trade-off \cite{Pasand.2017}}
\label{tab:RCS-tradeoff}
\resizebox{\linewidth}{!}{%
\begin{tabular}{|lccc|}
\hline
\rowcolor[HTML]{8DB4E1} 
\multicolumn{4}{|c|}{\cellcolor[HTML]{8DB4E1}\textbf{RCS Propellant Trade-off}}                                                                                         \\ \hline
\rowcolor[HTML]{C5D9F0} 
\multicolumn{1}{|l|}{\cellcolor[HTML]{C5D9F0}}             & \multicolumn{1}{c|}{\cellcolor[HTML]{C5D9F0}N2} & \multicolumn{1}{c|}{\cellcolor[HTML]{C5D9F0}N2} & Biprop \\ \hline
\multicolumn{1}{|l|}{Storage Temperature {[}K{]}}          & \multicolumn{1}{c|}{293.15}                     & \multicolumn{1}{c|}{77.00}                      & 293.15 \\ \hline
\multicolumn{1}{|l|}{RCS Engine Mass (8 engines) {[}kg{]}} & \multicolumn{1}{c|}{0.18}                       & \multicolumn{1}{c|}{0.18}                       & 5.46   \\ \hline
\multicolumn{1}{|l|}{RCS Propellant Mass {[}kg{]}}         & \multicolumn{1}{c|}{282.02}                     & \multicolumn{1}{c|}{189.00}                     & 93.66  \\ \hline
\multicolumn{1}{|l|}{RCS Tank Mass {[}kg{]}}               & \multicolumn{1}{c|}{190.02}                     & \multicolumn{1}{c|}{33.45}                      & 0.45   \\ \hline
\multicolumn{1}{|l|}{Total RCS System Mass {[}kg{]}}       & \multicolumn{1}{c|}{472.22}                     & \multicolumn{1}{c|}{222.63}                     & 99.57  \\ \hline
\end{tabular}}
\end{table}



\subsection{Arrangement Analysis}

In this section, the working principle of the arrangement analysis script (AAS) shall be described. The script itself is written in MATLAB and mainly used, to compute the center of gravity (CoG) and the moment of inertia (MoI) in the main axis frame of the spacecraft for each point in time. For that, a simplified model is set up. At first, the satellite shall be seen as a cuboid with a constant density distribution. All other parts with their known geometry, shape, location and mass are described as point-masses at a fixed location. This is demonstrated in Figure \ref{fig:araxy} in the XY-plane an Figure \ref{fig:arayz} in the Appendix \ref{Appendix}. To determine the location of the parts as well as the outer dimensions of the spacecraft itself, a feedback loop between the 3D-model and the AAS is used. The dry mass can then be split up into the known parts and into the rest, that is allocated to the cuboid. Because the spacecraft is loosing mass in form of propellant over each maneuver, the CoG is changing. To account for that, additional point masses are introduced, which represent the propellant. These masses start out in the center of their respective tank and move towards the bottom of the tank. While the total propellant mass and flow is taken from GMAT and the dV script, the flow of helium is a direct result of the pressure drop script. In this simulation, the movement of the CoG is limited to the X-direction. The result in the CoG shift is presented in Table \ref{tab:cog}. With this information, the solar panels and the subsystems, INS and COM, as well as the RCS system, can be placed at the location of the CoG before the $8^{th}$ maneuver. This location is chosen because it is assumed, that the RCS thrusters are mainly used in the quasi-satellite orbit phase for AOGNC maneuvers. The solar panel mass is determined with its area from the mass budget script and a conversion factor of $5.67 {kg}/{m^{2}}$ \cite{JPL.2015}. After this placement, the CoG has to be computed again. Since the solar panels are placed in the Y- and the INS and COM subsystem in the Z-direction, control over the systems MoI is possible. Therefore, it can be guaranteed, that the stability condition, presented in Equation \ref{eq:SC-stab} is satisfied. While the satellites cuboid uses the moment of inertia for a volumetric body, all other masses shall be taken into account with the parallel-axis theorem around the CoG \cite{Gross.2018}. The resulting MoI's are displayed in Table \ref{tab:inertia}. Finally, a moment equilibrium around the CoG between the misaligned and offset main engine and the RCS system can be computed. The launcher requirements, shown in Table \ref{tab:sc-launch} as well as the the stability conditions and the disturbance torques need to be satisfied.